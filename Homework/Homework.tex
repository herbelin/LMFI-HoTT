\documentclass{article}[6pt]%[landscape,twocolumn,letterpaper,8pt]
\usepackage[latin1]{inputenc}
\usepackage{amsfonts}
\usepackage{amsthm}
\usepackage{amssymb}
\usepackage{graphicx}
\usepackage{stmaryrd}
\usepackage{nccmath}
\usepackage{bussproofs}
\usepackage[english]{babel}
\usepackage{hyperref}
\selectlanguage{english} 
\usepackage[all]{xy}
\usepackage{array}
\usepackage{rotating}
\usepackage{tikz-cd}
\usepackage{fourier}

 \usepackage[landscape,twocolumn]{geometry}
\geometry{margin=0.4in}
%\usepackage{tikz-cd}

\usepackage{forest}
\usepackage{color}

\newcommand{\se}[1]{\medbreak \medbreak \section{#1}}
\newcommand{\sse}[1]{\medbreak \subsection{#1}}
\newcommand{\ssse}[1]{\subsubsection*{#1}}


\newcommand{\U}{{\mathcal U}}
\renewcommand{\r}{\rightarrow}
\newcommand{\Gl}{\lambda}

\newcommand{\ap}{\mathrm{ap}}
\newcommand{\apd}{\mathrm{apd}}
\newcommand{\refl}{\mathrm{refl}}
\newcommand{\id}{\mathrm{\bf id}}
\newcommand{\tr}{\mathrm{\bf tr}}
\newcommand{\fib}{\mathrm{\bf fib}}
\newcommand{\ua}{\mathrm{\bf ua}}
\newcommand{\base}{\mathrm{\bf base}}
\renewcommand{\loop}{\mathrm{\bf loop}}
\newcommand{\Map}{\mathrm{Map}}
\newcommand{\N}{\mathrm{\bf N}}
\renewcommand{\S}{\mathrm{\bf S}}
\newcommand{\merid}{\mathrm{\bf merid}}
\newcommand{\Grp}{\mathrm{Grp}}
\newcommand{\Hom}{\mathrm{Hom}}
\newcommand{\Iso}{\mathrm{Iso}}
\newcommand{\B}{\mathrm{B}}
\newcommand{\Aut}{\mathrm{Aut}}

\newcommand{\List}{\mathrm{\bf List}}

\newcommand{\s}{\mathrm{\bf s}}
\newcommand{\inc}{\mathrm{\bf inc}}

\newcommand{\one}{{\bf 1}}
\newcommand{\zero}{{\bf 0}}
\newcommand{\two}{{\bf 2}}
%\newcommand{\N}{\mathbb{N}}

\newcommand{\Set}{\mathrm{Set}}
\newcommand{\Prop}{\mathrm{Prop}}
\newcommand{\Gpd}{\mathrm{Gpd}}
\newcommand{\Eq}{\mathrm{Eq}}

\newcommand{\encode}{\mathrm{\bf encode}}
\newcommand{\decode}{\mathrm{\bf decode}}

\newtheorem{lemma}{Lemma}
\newtheorem{definition}{Definition}
\newtheorem{proposition}{Proposition}
\newtheorem{theorem}{Theorem}
\newtheorem{remark}{Remark}
\newtheorem{ind_def}{Inductive Definition}
\newtheorem{assumption}{Assumption}
\newtheorem{exercise}{Exercise}
\newtheorem{warning}{\danger Warning}
\newtheorem{open_problem}{Open problem}
%\newtheorem{example}{Example}

\newcommand{\lef}{\mathrm{\bf left}}
\newcommand{\righ}{\mathrm{\bf right}}
\newcommand{\quo}{\mathrm{\bf quo}}

\usepackage{exercise}



\title{Synthetic Homotopy Theory Homework: \\ The Hopf fibration}
\date{}

\begin{document}

\maketitle


%\begin{Exercise}[title={Yoneda Lemma}]
%TODO
%\end{Exercise}

The goal of this homework is to construct the Hopf fibration. We use notations from the lecture notes freely. First we need an auxiliary definition.

%There is a homework due on the 13th march, which will give bonus points for the exam. It is intended to be hard, so you should feel free to help each other and use the internet (e.g. section 8.5 in the HoTT book). The exercises are {\bf not} independent. Some questions are tagged as optional meaning they are a bit more tedious or technical than the others, but they are still needed to solve next questions.

\begin{definition}
A span consists $A,B,C:\U$ with $f:B\r A$ and $g:B\r C$.
\end{definition}

Usually we will denote a span by a diagram:
\[A \overset{f}{\leftarrow} B \overset{g}{\r} C\]
We give an analysis of equality between spans.

\begin{proposition}
Assume given two spans:
\[A \overset{f}{\leftarrow} B \overset{g}{\r} C \hspace{1.5cm} A' \overset{f'}{\leftarrow} B' \overset{g'}{\r} C'\]
then giving an equality between them is equivalent to giving three equivalences:
\[\epsilon_A : A \simeq A'  \hspace{1cm} \epsilon_B : B \simeq B'  \hspace{1cm} \epsilon_C : C \simeq C'\]
such that:
\[\epsilon_A \circ f \sim f'\circ \epsilon_B \hspace{1.5cm} \epsilon_C \circ g \sim g'\circ \epsilon_B\]

\end{proposition}

Now we introduce a new higher inductive type.

\begin{ind_def}
Assume given a span: 
\[A \overset{f}{\leftarrow} B \overset{g}{\r} C\]
\begin{itemize}
\item There is a type $A\coprod_B C$ called the pushout of the span.
\item For all $x:A$ we have:
\[\lef(x) : A\coprod_BC\] 
For all $y:C$ we have: 
\[\righ(y) : A\coprod_BC\] 
Moreover for all $z:B$ we have:
\[\quo(z) : \lef(f(z)) =_{A\coprod_BC} \righ(g(z))\]
\item Assume given $P : A\coprod_BC \r \U$, in order to define:
$f : (z:A\coprod_BC) \r P(z)$
it is enough to define:
\[f(\lef(x))\ :\equiv\ t_x\] 
with $t_x:P(\lef(x))$ for $x:A$, 
\[f(\righ(y))\ :\equiv\ s_y\] 
with $s_y:P(\righ(y))$ for $y:B$, and:
\[\apd_f(\quo(z)) \ :\equiv\  h_z\]
with $h_z : \tr^P_{\quo(z)} (t_{f(z)}) = s_{g(z)} $ for $z:B$.
\end{itemize}
\end{ind_def}


\begin{Exercise}[title={The join of two spaces}]

First we define the join of two spaces.

\begin{definition}
Assume given two types $A$ and $B$, then we define their join $A\ast B$ as the pushout of the span:
\[A\overset{p_1}{\leftarrow} A\times B \overset{p_2}{\r} B\]
where $p_1$ and $p_2$ are projections.%is the projection to $A$, and $p_2$ the second.
\end{definition}

%\paragraph{Question 1} Show that for any $A,B,C:\U$, we have:
%\[(A\ast B \r C)\ \simeq\ (f:A\r C)\times (g:B\r C)\times \big((x:A)\r (y:B) \r f(x)=_Cg(y)\big)\]

\paragraph{Question 1} Let $A$ be a type, show that:
\[\two\ast A \ \simeq \Sigma A\]

\paragraph{Question 2 (Optional)} Show that the join operation is associative, meaning that for all $A,B,C:\U$ we have:
\[(A\ast B)\ast C \ \simeq\ A\ast(B\ast C)\] 

\paragraph{Question 3} Using the two previous questions and the fact that $\Sigma S^n = S^{n+1}$, prove that:
\[S^1\ast S^1 \ \simeq \ S^3\]

\end{Exercise} 


\begin{Exercise}[title={Flattening Lemma for suspension}]
Assume given $A:\U$ and $P:\Sigma A \r \U$.

\paragraph{Question 1 (Optional)} Prove that $(x:\Sigma A)\times P(x)$ is equivalent to the pushout of the span:
\[P(\N) \overset{p_1}{\leftarrow} P(\N)\times A \overset{\psi}{\r} P(\S) \]
where $\psi$ is defined for $q:P(\N)$ and $a:A$ by:
\[\psi(q,a) \ :\equiv\ \tr^P_{\merid_a}(q)\]
\end{Exercise}



\begin{Exercise}[title={The Hopf construction}]
Assume given $A:\U$ with $\mu : A\times A \r A$ such that for all $a:A$ the maps
\[\mu(a,\_) \ \equiv\ \Gl x.\mu(a,x) : A\r A\]
\[\mu(\_,a) \ \equiv\ \Gl x.\mu(x,a):A\r A\] 
are equivalences.

We define $P:\Sigma A\r \U$ by:
\[P(\N) \ :\equiv\ A\]
\[P(\S) \ :\equiv\ A\]
and for $x:A$ we have:
\[\ap_P(\merid_x) \ :\equiv\ \ua^{-1}(\mu(\_,x))\]
where $\ua^{-1} : A\simeq B \r A=_\U B$ is the map assumed by univalence.

\paragraph{Question 1} Prove that for all $a,b:A$, we have $\tr^P_{\merid_b}(a) = \mu(a,b)$.

\paragraph{Question 2} Using the previous exercise, show that $(x:\Sigma)\times P(x)$ is equivalent to the pushout of the span:
\[ A \overset{p_1}{\leftarrow} A\times A \overset{\mu}{\r} A\]
%higher inductive type generated by:

\paragraph{Question 3} Using the fact that $\mu(a,\_)$ is an equivalence, show that the span:
\[ A \overset{p_1}{\leftarrow} A\times A \overset{\mu}{\r} A\]
is equal to the span:
\[ A \overset{p_1}{\leftarrow} A\times A \overset{p_2}{\r} A\]
where $p_1(x,y):\equiv x$ and $p_2(x,y):\equiv y$.

\paragraph{Question 4} Conclude that we have:
\[(x:\Sigma A) \times P(x) \ \simeq\ A\ast A\]

%\paragraph{Question 5} Conclude that with the assumption of this exercise, we have a fiber sequence

\end{Exercise}


\begin{Exercise}[title={$H$-types}]

In this exercise we define $H$-types and show that connected $H$-types satisfy the hypothesis from the previous exercise, and then build a fiber sequence.

\begin{definition}
An $H$-type consists of a type $A$ with:
\begin{itemize}
\item An element $e:A$.
\item A map: 
\[\mu : A\times A \r A\] 
such that for all $x:A$ we have:
\[\mu(x,e) = \mu(e,x)= x\]
\end{itemize} 
\end{definition}

As usual we identify an $H$-type and its underlying type. Let $A$ be a connected $H$-type.

\paragraph{Question 1} Show that for all $x:A$, we have $|\mu(x,\_) = \id_A|$ and $|\mu(\_,x) = \id_A|$.

\paragraph{Question 2} From the previous question, prove that for all $x:A$, the maps $\mu(x,\_)$ and $\mu(\_,x)$ are equivalences.

\vspace{0.4cm}

Recall that given $X$ a pointed type and $C:X\r \U$ with $*_C:C(*)$, we can build a fiber sequence:
\[C(*) \r_* (x:X)\times C(x) \r_* X\]

\paragraph{Question 3} Using the previous exercise, show that we have a fiber sequence:
\[A\r_* A\ast A\r_*\Sigma A\] 

\end{Exercise}


\begin{Exercise}[title={$S^1$ is a $H$-type}]

In this exercise we build the Hopf fibration using the fact that $S^1$ is a $H$-type.

\paragraph{Question 1 (Optional)} Define: 
\[\psi' : (x:S^1)\r x=_S^1x\]
with $\psi'(\base) :\equiv \loop$.

\vspace{0.4cm}

We define $\mu : S^1\r  S^1\r S^1$ by:
\[\mu(\base) \ :\equiv\ \id_{S^1} : S^1\r S^1\]
\[\ap_\mu(\loop) \ :\equiv\ \psi\]

where $\psi : \id_{S^1}=\id_{S^1}$ is the image of $\psi'$ by function extensionality.

\paragraph{Question 2 (Optional)}
Prove that for all $x:S^1$ we have:
\[\mu(x,\base) = x\]
\[\mu(\base,x) = x\]

\paragraph{Question 3} Using the previous exercises, conclude that we have a fiber sequence:
\[S^1\r_* S^3\r_* S^2\]

\paragraph{Question 4} Conclude that:
\[\pi_n(S^3) = \pi_n(S^2)\]
if $n>2$.

\end{Exercise}


\end{document}


