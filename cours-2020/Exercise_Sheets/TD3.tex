\documentclass{article}[6pt]%[landscape,twocolumn,letterpaper,8pt]
\usepackage[latin1]{inputenc}
\usepackage{amsfonts}
\usepackage{amsthm}
\usepackage{amssymb}
\usepackage{graphicx}
\usepackage{stmaryrd}
\usepackage{nccmath}
\usepackage{bussproofs}
\usepackage[english]{babel}
\usepackage{hyperref}
\selectlanguage{english} 
\usepackage[all]{xy}
\usepackage{array}
\usepackage{rotating}
\usepackage{tikz-cd}
\usepackage{fourier}

 \usepackage[landscape,twocolumn]{geometry}
\geometry{margin=0.4in}
%\usepackage{tikz-cd}

\usepackage{forest}
\usepackage{color}

\newcommand{\se}[1]{\medbreak \medbreak \section{#1}}
\newcommand{\sse}[1]{\medbreak \subsection{#1}}
\newcommand{\ssse}[1]{\subsubsection*{#1}}


\newcommand{\U}{{\mathcal U}}
\renewcommand{\r}{\rightarrow}
\newcommand{\Gl}{\lambda}
\newcommand{\comment}[1]{}

\newcommand{\ap}{\mathrm{ap}}
\newcommand{\apd}{\mathrm{apd}}
\newcommand{\refl}{\mathrm{refl}}
\newcommand{\id}{\mathrm{\bf id}}
\newcommand{\tr}{\mathrm{\bf tr}}
\newcommand{\fib}{\mathrm{\bf fib}}
\newcommand{\ua}{\mathrm{\bf ua}}
\newcommand{\base}{\mathrm{\bf base}}
\renewcommand{\loop}{\mathrm{\bf loop}}
\newcommand{\Map}{\mathrm{Map}}
\newcommand{\N}{\mathrm{\bf N}}
\renewcommand{\S}{\mathrm{\bf S}}
\newcommand{\merid}{\mathrm{\bf merid}}
\newcommand{\Grp}{\mathrm{Grp}}
\newcommand{\Hom}{\mathrm{Hom}}
\newcommand{\Iso}{\mathrm{Iso}}
\newcommand{\B}{\mathrm{B}}
\newcommand{\Aut}{\mathrm{Aut}}

\newcommand{\List}{\mathrm{\bf List}}

\newcommand{\s}{\mathrm{\bf s}}
\newcommand{\inc}{\mathrm{\bf inc}}

\newcommand{\one}{{\bf 1}}
\newcommand{\zero}{{\bf 0}}
\newcommand{\two}{{\bf 2}}
%\newcommand{\N}{\mathbb{N}}

\newcommand{\Set}{\mathrm{Set}}
\newcommand{\Prop}{\mathrm{Prop}}
\newcommand{\Gpd}{\mathrm{Gpd}}
\newcommand{\Eq}{\mathrm{Eq}}

\newcommand{\encode}{\mathrm{\bf encode}}
\newcommand{\decode}{\mathrm{\bf decode}}

\newtheorem{lemma}{Lemma}
\newtheorem{definition}{Definition}
\newtheorem{proposition}{Proposition}
\newtheorem{theorem}{Theorem}
\newtheorem{remark}{Remark}
\newtheorem{ind_def}{Inductive Definition}
\newtheorem{assumption}{Assumption}
\newtheorem{exercise}{Exercise}
\newtheorem{warning}{\danger Warning}
\newtheorem{open_problem}{Open problem}
%\newtheorem{example}{Example}

\usepackage{exercise}



\title{Synthetic Homotopy Theory TD3: \\ Homotopy groups}

\begin{document}

\maketitle


\begin{Exercise}[title={The homotopy groups of $S^1$}]

%\paragraph{Question 1} Show that $S^1\simeq \Sigma S^0$.

\paragraph{Question 1} Compute the homotopy groups of $S^1$. 

\paragraph{Remark} The same problem for $S^2$ is open.
\end{Exercise}

\begin{Exercise}[title={Eckmann-Hilton argument}]
In this exercise we want to show that $|\Omega^2 X|_0$ has a structure of {\bf abelian} group for any pointed type $X$.% A unital binary operation 

\paragraph{Question 1} Assume given a type $X:\U$ with two binary operations: 
\[\_\cdot\_ \, :\, X\r X\r X\]
\[\_\otimes\_ \, :\,  X\r X\r X\] 
with the same unit $e:X$, meaning that for all $x:X$ we have: 
\[x\cdot e = e\cdot x = x\]
\[x\otimes e = e\otimes x = x\]
Moreover assume that for any $a,b,c,d:X$ we have:
\[(a\otimes b)\cdot(c\otimes d) = (a\cdot c)\otimes(b\cdot d)\]
Prove that $\_\cdot\_$ and $\_\otimes\_$ are equal and that they both commute.

%\paragrapf{Question 2 (optional)}

\paragraph{Question 2} Assume given $A:\U$ with $p,q : x=_Ay$ and $p',q':y=_Az$. Moreover assume given $h:p=q$ and $h':p'=q'$. Define:
\[h\otimes h': p\cdot q = p'\cdot q'\]

\paragraph{Question 3} Assume given a type $X$ with $x:X$. Prove that the composition of paths $\_\cdot\_$ and the operation $\_\otimes\_$ from the previous question induce operations on: 
\[\Omega^2 X \ :\equiv\ \refl_x=_{x=_Xx}\refl_x\]
obeying the hypothesis from question 1.

\paragraph{Question 4} Conclude that the canonical group structure on $|\Omega^2 X|_0$ is abelian.

\end{Exercise}


\begin{Exercise}[title={Loop and suspension}]
Recall that we defined $\Sigma : \U_* \r \U_*$ and $\Omega:\U_*\r\U_*$ by:
\[\Sigma X \ :\equiv\ (\Sigma X,\N)\]
\[\Omega X\ :\equiv\ (*=_X*,\refl_*)\]
In this exercise we want to show that for any $X,Y:\U_*$ we have:
\[(\Sigma X \r_* Y)\ \simeq\ (X\r_*\Omega Y)\]

\paragraph{Question 1} Assume given $f:\Sigma X\r_* Y$ (recall this means we have $*_f:f(\N)=*$). We define $\psi(f) : X\r \Omega Y$ by:
\[\psi(f,x) \ :\equiv\ *_f^{-1} \cdot \ap_f(\merid_x\cdot\merid_*^{-1})\cdot *_f\]
Show this actually defines a map:
\[\psi : (\Sigma X \r_* Y) \r (X\r_*\Omega Y)\]

\paragraph{Question 2} Assume given $g:X\r_*\Omega Y$, we define $\phi(g) : \Sigma X \r Y$ by:
\[\phi(g,\N) \ :\equiv\ *\]
\[\phi(g,\S) \ :\equiv\ *\]
\[\ap_{\phi(g)}\merid_x \ :\equiv\ f(x)\]
Show this actually defines a map:
\[\phi :  (X\r_*\Omega Y) \r (\Sigma X \r_* Y)\]

\paragraph{Question 3} Show that given $f,g:\Sigma X \r Y$, in order to prove $f=g$ it is enough to give:
\[p:f(\N)=g(\N)\]
\[q:f(\S)=g(\S)\]
\[h : (x:X) \r \ap_f(\merid_x) \cdot q = p\cdot \ap_g(\merid_x)\]
Using this show that for all $f:\Sigma X \r_* Y$ we have: 
\[\phi(\psi(f)) =_{\Sigma X \r Y} f\]

\paragraph{Question 4 (Optional)} Can you prove that:
\[\phi(\psi(f)) =_{\Sigma X \r_* Y} f\]
(Can you understand what you need to prove? Hint: that paths between basepoints agree).

\paragraph{Question 5} Show that: 
\[\psi(\phi(g)) =_{X\r\Omega Y} g\] 
for all $g:X\r_*\Omega Y$.

\paragraph{Question 6 (Optional)} Try to prove:
\[\psi(\phi(g)) =_{X\r_*\Omega Y} g\]
(Can you understand what you need to prove? Hint: that paths between basepoints agree).

\paragraph{Question 7} Conclude from the previous questions that:
\[(\Sigma X \r_* Y)\ \simeq\ (X\r_*\Omega Y)\]

\end{Exercise}


\begin{Exercise}[title={Canonical fiber sequence},difficulty=1]

\paragraph{Question 1} Prove that any fiber sequence is equal to a canonical one.

\paragraph{Question 2} Show that giving $Y:\U_*$ and a map in $Y\r_* Z$ is the same as giving $P:X\r \U$ and $*_P:P(*_Z)$.

\paragraph{Question 3} Using previous questions, prove that any fiber sequence is of the form:
 \[P(*_Z) \overset{\inc}\r_* (z:Z)\times P(z)\overset{p_Z}{\r_*} Z\]
 where $P(*_Z$) is pointed by $*_P$ and $(z:Z)\times P(z)$ is pointed by $(*_Z,*_P)$, with $p_Z$ the projection and $\inc(q):\equiv (*_Z,q)$.
\end{Exercise}



\begin{Exercise}[title={Representable invariants}]
A representable invariant (implicitly: on based types) is a map $F : \U_* \r \Set_*$ of the form:
\[\Gl (X:\U_*).\, |A\r_* X|_0 \]
for some fixed $A$, with $|A\r_*X|_0$ pointed by the constant map. 

\paragraph{Question 1} Show that homotopy groups are representable invariants.
\vspace{0.6cm}

\noindent We define the product of two pointed types $X,Y:\U_*$ as $X\times Y$ pointed by $(*_X,*_Y)$.
\paragraph{Question 2} Show that:
\[F(X\times Y) = F(X)\times F(Y)\]
for any representable invariant $F$.
\vspace{0.4cm}

\noindent A family of pointed types is a map $X:I\r \U_*$ for $I$ a set. We denote such a family by $(X_i)_{i:I}$, with $X_i$ denoting the element $X(i)$. 
The product $\Pi_{i:I}X_i$ of a family of type is defined as the type $(i:I)\r X(i)$ pointed by $\Gl (i:I).*_{X(i)}$.

\paragraph{Question 3}  Using an appropriate version of the axiom of choice, prove that: 
\[F(\Pi_{i:I}X_i) \ = \ (i:I)\r F(X_i)\]  

\paragraph{Question 4 (Optional)} Let $X\r_* Y\r_* Z$ be a fiber sequence, prove that we have an exact sequence of pointed set:
\[F(X)\r_* F(Y)\r_* F(Z)\]
for any representable invariant $F$. (Hint: use the "canonical fiber sequence" exercise)

\end{Exercise}


\comment{
\begin{Exercise}[title={Corepresentable invariants},difficulty=1]
This exercise is in some sense dual to the previous one. A corepresentable invariant (implicitly: on based types) is a map $G : \U_* \r \Set_*$ of the form:
\[\Gl (X:\U_*).\, |X\r_* A|_0 \]
for some fixed $A$, with $|X\r_*A|_0$ pointed by the constant map. 

%\paragraph{Question 1} Show that homotopy groups are representable invariants.

We define the wedge $X\wedge Y$ of two pointed types $X,Y:\U_*$ as a higher inductive type.

\begin{ind_def}
TODO
\end{ind_def}

\paragraph{Question 1} Show that: 
\[G(X\wedge Y) = G(X)\times G(Y)\]
for any corepresentable invariant $G$.
\medbreak
Now we define the wedge $\wedge_{i:I}X_i$ of a family of pointed type.

\begin{ind_def}
TODO
\end{ind_def}

%A family of pointed types is a map $X:I\r \U_*$ for $I$ a set. We denote such a family by $(X_i)_{i:I}$, with $X_i$ denoting the element $X(i)$. 
%The product $\Pi_{i:I}X_i$ of a family of type is defined as the type $(i:I)\r X(i)$ pointed by $\Gl (i:I).*_{X(i)}$.

\paragraph{Question 3}  Using an appropriate version of the axiom of choice, prove that: 
\[G(\wedge_{i:I}X_i) \ = \ (i:I)\r G(X_i)\]  
for any corepresentable invariant $G$.
\vspace{0.6cm}

\noindent TODO define cofiber sequence.

\paragraph{Question 4} Let $X\r_* Y\r_* Z$ be a cofiber sequence, prove that we have an exact sequence of pointed set:
\[G(Z)\r_* G(Y)\r_* G(X)\]
for any corepresentable invariant $G$ (beware: the arrow are reversed).

\end{Exercise}
}


\begin{Exercise}[title={The long fiber sequence of a map}]
 We want to prove that given a fiber sequence:
 \[X\overset{f}{\r_*} Y\overset{g}{\r_*} Z\] 
 we have a long fiber sequence:
 \[ \cdots \r_*\Omega X \r_* \Omega Y \r_* \Omega Z \r_* X\r_* Y\r_* Z\]
 
 \paragraph{Question 1} Show that given a fiber sequence:
  \[X\r_*Y\r_*Z\] 
  we have a fiber sequence:
   \[\Omega Z\r_*X\r_* Y\] 
(Hint: use the "canonical fiber sequence" exercise. The necessary map from $\Omega Z$ to $P(*_Z)$ sends $r$ to $\tr_r^P(*_P)$).
  
 
% \paragraph{Question 2 (Optional)} Show that giving $Y:\U_*$ and a map in $Y\r_* Z$ is the same thing as giving $P:X\r \U$ and $*_P:P(*_Z)$. Using this equivalence and the fact that any fiber sequence is equal to a canonical one, prove that any fiber sequence is of the form:
% \[P(*_Z) \overset{\inc}\r_* (x:Z)\times P(x)\]
%  where $P(_Z*$) is pointed by $*_P$ and $(z:Z)\times P(x)$ is pointed by $(*_Z,*_P)$, with $p_Z$ the projection and $\inc(q):\equiv (*_Z,q)$.
 
 \paragraph{Question 2} Conclude by iterating the previous question.
 
% \paragraph{Question 4} Conclude
 
 \paragraph{Question 3 (Optional)} Can you build a long fiber sequence:
\[\cdots \r_* \Omega^2 X \overset{\Omega^2 f}{\r_*} \Omega^2 Y \overset{\Omega^2 g}{\r_*}\Omega^2 Z \overset{\Omega\delta}{\r_*} \Omega X \overset{\Omega f}{\r_*} \Omega Y \overset{\Omega g}{\r_*} \Omega Z \overset{\delta}{\r_*} X \overset{f}{\r_*} Y\overset{g}{\r_*} Z  \]
 (you need to be careful about maps).
 
\end{Exercise}



\end{document}


