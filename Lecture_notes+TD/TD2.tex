\documentclass{article}[6pt]%[landscape,twocolumn,letterpaper,8pt]
\usepackage[latin1]{inputenc}
\usepackage{amsfonts}
\usepackage{amsthm}
\usepackage{amssymb}
\usepackage{graphicx}
\usepackage{stmaryrd}
\usepackage{nccmath}
\usepackage{bussproofs}
\usepackage[english]{babel}
\usepackage{hyperref}
\selectlanguage{english} 
\usepackage[all]{xy}
\usepackage{array}
\usepackage{rotating}
\usepackage{tikz-cd}
\usepackage{fourier}

 \usepackage[landscape,twocolumn]{geometry}
\geometry{margin=0.4in}
%\usepackage{tikz-cd}

\usepackage{forest}
\usepackage{color}

\newcommand{\se}[1]{\medbreak \medbreak \section{#1}}
\newcommand{\sse}[1]{\medbreak \subsection{#1}}
\newcommand{\ssse}[1]{\subsubsection*{#1}}


\newcommand{\U}{{\mathcal U}}
\renewcommand{\r}{\rightarrow}
\newcommand{\Gl}{\lambda}

\newcommand{\ap}{\mathrm{ap}}
\newcommand{\apd}{\mathrm{apd}}
\newcommand{\refl}{\mathrm{refl}}
\newcommand{\id}{\mathrm{\bf id}}
\newcommand{\tr}{\mathrm{\bf tr}}
\newcommand{\fib}{\mathrm{\bf fib}}
\newcommand{\ua}{\mathrm{\bf ua}}
\newcommand{\base}{\mathrm{\bf base}}
\renewcommand{\loop}{\mathrm{\bf loop}}
\newcommand{\Map}{\mathrm{Map}}
\newcommand{\N}{\mathrm{\bf N}}
\renewcommand{\S}{\mathrm{\bf S}}
\newcommand{\merid}{\mathrm{\bf merid}}
\newcommand{\Grp}{\mathrm{Grp}}
\newcommand{\Hom}{\mathrm{Hom}}
\newcommand{\Iso}{\mathrm{Iso}}
\newcommand{\B}{\mathrm{B}}
\newcommand{\Aut}{\mathrm{Aut}}

\newcommand{\List}{\mathrm{\bf List}}

\newcommand{\s}{\mathrm{\bf s}}
\newcommand{\inc}{\mathrm{\bf inc}}

\newcommand{\one}{{\bf 1}}
\newcommand{\zero}{{\bf 0}}
\newcommand{\two}{{\bf 2}}
%\newcommand{\N}{\mathbb{N}}

\newcommand{\Set}{\mathrm{Set}}
\newcommand{\Prop}{\mathrm{Prop}}
\newcommand{\Gpd}{\mathrm{Gpd}}
\newcommand{\Eq}{\mathrm{Eq}}

\newcommand{\encode}{\mathrm{\bf encode}}
\newcommand{\decode}{\mathrm{\bf decode}}

\newtheorem{lemma}{Lemma}
\newtheorem{definition}{Definition}
\newtheorem{proposition}{Proposition}
\newtheorem{theorem}{Theorem}
\newtheorem{remark}{Remark}
\newtheorem{ind_def}{Inductive Definition}
\newtheorem{assumption}{Assumption}
\newtheorem{exercise}{Exercise}
\newtheorem{warning}{\danger Warning}
\newtheorem{open_problem}{Open problem}
%\newtheorem{example}{Example}

\usepackage{exercise}



\title{Synthetic Homotopy Theory TD2: Higher inductive types, truncations and logic}

\begin{document}

\maketitle

\begin{Exercise}[title={Higher inductive types}]

\paragraph{Question 1} Let $A$ be a type, show that: 
\[(S^1\r A)\ \simeq\ (x:A)\times (x=x)\]

\paragraph{Question 2} Prove that $\Sigma \two \ \simeq \ S^1$. 

\paragraph{Question 3 (Optional)} Prove that $S^1\times S^1 \simeq T$. Prove that $\Sigma S^1 \ \simeq \ S^2$.
%\paragraph{Question 4}
%Assume $X$ and $Y$ are pointed types. ????% be a pointed type
\end{Exercise}


%\begin{Exercise}[title={Suspensions}]
%Recall that a pointed type is a type $X$ with an inhabitant $x:X$. The type of pointed type is denoted $$
%\end{Exercise}

\begin{Exercise}
In this exercise we use results from TD1 to prove truncation results for types.

\paragraph{Question 1} Show that $\one$ is contractible. %and $\zero$ are propositions.

\paragraph{Question 2} Show that $\zero$ is a proposition.

\paragraph{Question 3} Show that $\two$ and $\mathbb{N}$ are sets. Show that they are not propositions.

\paragraph{Question 4} Let $A$ be a proposition. Show that $A+ \lnot A$ is a proposition. Find $A,B:\Prop$ such that $A+B$ is not a proposition (Hint: recall that $\one+\one = \two$).

\paragraph{Question 5} Show that the type of sets is a groupoid.

\paragraph{Question 6 (Optional)} Show that the type of contractible types is contractible.

\end{Exercise}


\begin{Exercise}[title={Identity types in set-truncations}]
\paragraph{Question 1} Let $A$ be a type, using the encode-decode method show that: 
\[[x]=_{|A|_0}[y]\ \simeq \ |x=_A y|\]
\end{Exercise}


\begin{Exercise}[title={Decidable types and propositions}]

%We give two definitions.

\begin{definition}
A proposition $A$ is called decidable if we can prove $A+\lnot A$. A type $X$ is said to have decidable equality if for all $x,y:X$ the type $(x=y)+\lnot(x=y)$ is provable.
\end{definition}

\paragraph{Question 1} Let $A$ be a decidable proposition, prove that:
\[(A=\zero) + (A=\one)\]

\paragraph{Question 2} Show that types with decidable equality are sets (Hint: use the encode-decode method to compute the identity types of a decidable type).

%\paragraph{Question 3}  Show that not all types are decidable (Hint: $$).
\begin{remark}
It is quite natural to ask whether all propositions are decidable. This cannot be proved or disproved in HoTT.
\end{remark}

\end{Exercise}


\begin{Exercise}[title={The law of the excluded middle}]
In this exercise we consider the law of the excluded middle in type theory. The naive law of the excluded middle is an inhabitant of the type $(A:\U)\r A+\lnot A$.
%\[(A:\U)\r A+\lnot A\]

\paragraph{Question 1} Show that univalence contradicts the naive law of the excluded middle.

\begin{definition}
The law of excluded middle is the following type:
\[(A:\Prop)\r A + \lnot A\]
Equivalently it says that all propositions are decidable.
\end{definition}

The law of excluded cannot be proved nor disproved in HoTT. Its interpretation in homotopy types is true. 

\paragraph{Question 2} Assuming the law of the excluded middle, show that $\Prop =_\U \two$.

\end{Exercise}


\begin{Exercise}[title={Propositional truncation}]

\paragraph{Question 1} Let $A$ be a type. Show that the map $[\_] : A\r |A|$ induces equivalences:
\[\Gl f,x.\, f([x]) : (|A|\r B) \r (A\r B)\]
for all proposition $B$.%, définit par $\psi(f) = f \circ [\_]$.

%\paragraph{Question 2} Let $A$ be a type 

\paragraph{Question 2} Let $A$ be a type. Assume given a type $C$ with a map $t:A\r C$ such that we have equivalences:
\[\Gl f.\, f\circ t:(C\r B) \r (A\r B)\]
for all proposition $B$. Show that $C\simeq |A|$.

\paragraph{Question 3} Assuming the law of excluded middle, prove that $\lnot\lnot A$ is equivalent to $|A|$.   

\end{Exercise}


\begin{Exercise}[title={Connected types}]
Recall that a type $X$ is said connected if $(x,y:X)\r |x=y|$.

\paragraph{Question 1} Let $X$ be a connected type with $x:X$. Show that $|X|_0=\one$.

\paragraph{Question 2} Let $X$ be a type such that $|X|_0=\one$, show that $X$ is connected.

\paragraph{Question 3} Show that the circle is connected.

\paragraph{Question 4} Assume given $X:\U$ with $x:X$. Show that $\Sigma X$ is connected.

\end{Exercise}


\begin{Exercise}[title={The axiom of choice}]

\begin{definition}
The naive axiom of choice says that for any types $A,B:\U$ and $P:A\r B\r \U$ we have an inhabitant of: %is the type: 
\[\big((x:A)\r (y:B)\times P(x,y) \big) \r (f:A\r B) \times \big((x:A)\r P(x,f(x))\big)\]
\end{definition}

\paragraph{Question 1} Prove the naive axiom of choice.

\begin{definition}
The axiom of choice says that for all {\bf set} $A$ with $P:A\r \U$ we have:
\[ \big((x:A)\r |P(x)|\big) \r |(x:A)\r P(x)|\]
\end{definition}

\begin{remark}
There exists a lot of variants of this axiom in HoTT, for example $P$ can be a family of sets, or we can use set-truncation rather than propositional truncation, etc.
\end{remark}

\paragraph{Question 2 (Optional)} Prove that the same axiom for all $A:\U$ contradicts univalence.

\end{Exercise}


\begin{Exercise}[title={Diaconescu's theorem}]
In this exercise we show that the axiom of choice imply the law of the excluded middle. Assume given $A:\Prop$.

\paragraph{Question 1} Using the encode-decode method, prove that $\Sigma A$ is a set and that: 
\[(\N=_{\Sigma A} \S) \ \simeq\ A\]

\paragraph{Question 2} We define $f:\two \r \Sigma A$ by $f(0):\equiv \N$ and $f(1):\equiv \S$. Show that:
\[(x:\Sigma A) \r |\fib_f(x)|\]

\paragraph{Question 3} Using the axiom of choice, conclude that:
\[|(g:\Sigma A \r \two) \times (f\circ g \sim \id)|\]

\paragraph{Question 4} Assume given $g:\Sigma A \r \two$ such that $f\circ g \sim \id$. Show that:
\[(g(\N)=_\two g(\S))+(g(\N){\not=}_\two g(\S))\]
 Show that $g(\N)=g(\S)\r \N=\S$ and $g(\N)\not=g(\S) \r \N\not=\S$.

%\paragraph{Question 4} Prove that we have $A+\lnot A$ using $g$ (Hint: recall that $A+\lnot A$ is a proposition, and then notice that $g(\N)=g(\S)\r \N = \S$ and $g(\N)\not=g(\S) \r \N\not=\S$). 

\paragraph{Question 5} From the previous questions, conclude that the axiom of choice implies the law of the excluded-middle.

\end{Exercise}






\end{document}


