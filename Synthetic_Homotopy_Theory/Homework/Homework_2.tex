\documentclass{article}[6pt]%[landscape,twocolumn,letterpaper,8pt]
\usepackage[latin1]{inputenc}
\usepackage{amsfonts}
\usepackage{amsthm}
\usepackage{amssymb}
\usepackage{graphicx}
\usepackage{stmaryrd}
\usepackage{nccmath}
\usepackage{bussproofs}
\usepackage[english]{babel}
\usepackage{hyperref}
\selectlanguage{english} 
\usepackage[all]{xy}
\usepackage{array}
\usepackage{rotating}
\usepackage{tikz-cd}
\usepackage{fourier}

 \usepackage[landscape,twocolumn]{geometry}
\geometry{margin=0.4in}
%\usepackage{tikz-cd}

\usepackage{forest}
\usepackage{color}

\newcommand{\se}[1]{\medbreak \medbreak \section{#1}}
\newcommand{\sse}[1]{\medbreak \subsection{#1}}
\newcommand{\ssse}[1]{\subsubsection*{#1}}


\newcommand{\U}{{\mathcal U}}
\renewcommand{\r}{\rightarrow}
\newcommand{\Gl}{\lambda}

\newcommand{\ap}{\mathrm{ap}}
\newcommand{\apd}{\mathrm{apd}}
\newcommand{\refl}{\mathrm{refl}}
\newcommand{\id}{\mathrm{\bf id}}
\newcommand{\tr}{\mathrm{\bf tr}}
\newcommand{\fib}{\mathrm{\bf fib}}
\newcommand{\ua}{\mathrm{\bf ua}}
\newcommand{\base}{\mathrm{\bf base}}
\renewcommand{\loop}{\mathrm{\bf loop}}
\newcommand{\Map}{\mathrm{Map}}
\newcommand{\N}{\mathrm{\bf N}}
\renewcommand{\S}{\mathrm{\bf S}}
\newcommand{\merid}{\mathrm{\bf merid}}
\newcommand{\Grp}{\mathrm{Grp}}
\newcommand{\Hom}{\mathrm{Hom}}
\newcommand{\Iso}{\mathrm{Iso}}
\newcommand{\B}{\mathrm{B}}
\newcommand{\Aut}{\mathrm{Aut}}

\newcommand{\List}{\mathrm{\bf List}}

\newcommand{\s}{\mathrm{\bf s}}
\newcommand{\inc}{\mathrm{\bf inc}}

\newcommand{\one}{{\bf 1}}
\newcommand{\zero}{{\bf 0}}
\newcommand{\two}{{\bf 2}}
%\newcommand{\N}{\mathbb{N}}

\newcommand{\Set}{\mathrm{Set}}
\newcommand{\Prop}{\mathrm{Prop}}
\newcommand{\Gpd}{\mathrm{Gpd}}
\newcommand{\Eq}{\mathrm{Eq}}

\newcommand{\encode}{\mathrm{\bf encode}}
\newcommand{\decode}{\mathrm{\bf decode}}

\newtheorem{lemma}{Lemma}
\newtheorem{definition}{Definition}
\newtheorem{proposition}{Proposition}
\newtheorem{theorem}{Theorem}
\newtheorem{remark}{Remark}
\newtheorem{ind_def}{Inductive Definition}
\newtheorem{assumption}{Assumption}
\newtheorem{exercise}{Exercise}
\newtheorem{warning}{\danger Warning}
\newtheorem{open_problem}{Open problem}
\newtheorem{goal}{Goal}
%\newtheorem{example}{Example}

\newcommand{\lef}{\mathrm{\bf left}}
\newcommand{\righ}{\mathrm{\bf right}}
\newcommand{\quo}{\mathrm{\bf quo}}

\usepackage{exercise}



\title{Synthetic Homotopy Theory Homework: \\ The Impredicative Circle}
\date{}

\begin{document}

\maketitle

\begin{definition} 
A map $f : A\rightarrow B$ is called an immersion if for all $x,y:A$, the map:
\[\mathrm{\bf ap}_f : x =_A y \rightarrow f(x) =_B f(y)\]
is an equivalence.
\end{definition}

\begin{goal} 
In this exercise we want to construct an immersion of $S^1$ into a type built without higher inductive type.
\end{goal}

\paragraph{Question 1} 
Let $A$ be a pointed connected type, and let $B$ be any type. Assume given $f:A\rightarrow B$ such that: 
\[\mathrm{\bf ap}_f : * = * \rightarrow f(*) = f(*)\]
is an equivalence (here $*$ is the basepoint of $A$). Prove that $f$ is an immersion.

\paragraph{Question 2} 
Assume given $B$ a type with $b:B$ such that we have an isomorphism of group:
\[\epsilon :  \mathbb{Z} \rightarrow (b=_Bb)\] 
We define $f : S^1 \rightarrow B$ by: 
\[f(\mathrm{\bf base})\ :\equiv\ b\]
\[\mathrm{\bf ap}_f(\mathrm{\bf loop})\ :\equiv\ \epsilon(1)\]
Using question 1 show that $f$ is an immersion.

\begin{definition} 
We denote by $\mathrm{Aut}$ the type $(X:{\mathcal U}) \times (X =_{\mathcal U} X)$.
\end{definition}

\paragraph{Question 3} 
Assume given $(X,p)$ and $(Y,q)$ in $\mathrm{Aut}$. Prove that:
\[(X,p) =_{\mathrm{Aut}} (Y,q) \ \simeq\ (\epsilon : X=_{\mathcal U}Y) \times (p\cdot\epsilon = \epsilon\cdot q)\]

\begin{definition}
We denote by $G$ the type of $\epsilon : \mathbb{Z}\simeq \mathbb{Z}$ such that we have the following property:
\[(n:\mathbb{Z}) \rightarrow \epsilon(n+1) = \epsilon(n)+1\]
\end{definition}

\paragraph{Question 4} 
We denote by $s$ the equivalence in $\mathbb{Z} \simeq \mathbb{Z}$ with underlying map $x\mapsto x+1$. Using question 3 prove that:
\[\big((\mathbb{Z},s) =_{\mathrm{Aut}} (\mathbb{Z},s)\big) \ =_{\mathcal U}\ G\]

\paragraph{Question 5} 
Show that $G$ has a group structure where multiplication is the composition of equivalences. We admit that the equality from question 4 induces an isomorphism of group.

\paragraph{Question 6} 
Show that we have an isomorphism of group:
\[\epsilon \mapsto \epsilon(0) : G \rightarrow \mathbb{Z}\]

\paragraph{Question 7} 
Conclude that we have an immersion of $S^1$ into $\mathrm{Aut}$.


% \ \simeq\ (\epsilon : \mathbb{Z}\simeq\mathbb{Z})\times\big((n:\mathbb{Z}) \rightarrow \epsilon(n+1) = \epsilon(n)+1\big)\]

\paragraph{Bonus Question} Show that if $A$ is a pointed-connected type and $B$ is any type, an immersion $f: A \rightarrow B$ induces an equivalence: 
\[A\ \simeq\ (b:B)\times | f(*) = b |\]

\begin{remark} 
We define:
\[T\ :\equiv\ (X:{\mathcal U})\times (p:X=_{\mathcal U} X) \times |(\mathbb{Z},s) = (X,p)|\]
From question 7 and the bonus question, we can deduce that:
\[S^1\ \simeq\ T\]
In fact we can define the circle as $T$. This means that the circle can be defined from propositional truncation, without using higher inductive type.

It should be noted that $T$ is one universe higher than the circle. % making it akin to an impredicative encoding of the circle. 
\end{remark}

\begin{remark}
By our correspondance between groups and connected pointed groupoids, it is natural to see the type $T$ from the previous remark as the type of $\mathbb{Z}$-torsors (see \url{https://ncatlab.org/nlab/show/torsor} for a definition of torsors, and \url{http://math.ucr.edu/home/baez/torsors.html} for an entertaining introduction). %So we can see $\mathbb{Z}$-torsors as an impredicative encoding for $S^1$.
%$\mathrm{B}\mathbb{Z}$ (recall that $S^1 = \mathrm{B}\mathbb{Z}$). 

This can be generalized from $\mathbb{Z}$ to any group $G$, by giving a definition (without using higher inductive) of the type of $G$-torsors , and proving that it is equivalent to $\mathrm{B}G$ (recall that $S^1 = \mathrm{B}\mathbb{Z}$). This gives an insight into the classical definition of $G$-torsors: they form a large type equivalent to $\mathrm{B} G$.
\end{remark}


\end{document}


